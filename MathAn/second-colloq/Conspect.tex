\documentclass{article}
\usepackage[utf8]{inputenc}
\usepackage[T2A]{fontenc}
\usepackage[russian]{babel}
\usepackage[normalem]{ulem}
\usepackage{amsfonts}
\usepackage{amsmath}
\usepackage{amsthm}
\usepackage{amssymb}
\usepackage{arcs}
\usepackage{fancyhdr}
\usepackage{float}
\usepackage[left=2cm,right=2cm,top=2cm,bottom=2cm]{geometry}
\usepackage{graphicx}
\usepackage{hyperref}
\usepackage{multicol}
\usepackage{stackrel}
\usepackage{xcolor}
\usepackage{cancel}

\makeatother
\makeatletter

\title{\textbf{Конспект лекций (12-...)}}
\author{Илья Астафьев}

\DeclareMathOperator*\lowlim{\underline{lim}}
\DeclareMathOperator*\uplim{\overline{lim}}

\newcommand*{\limToInf}[2]{\displaystyle \lim_{#1 \to \infty} #2}
\newcommand*{\limToZero}[2]{\displaystyle \lim_{#1 \to 0} #2}
\newcommand*{\lemma}[1]{\textbf{Лемма.} #1. \newline}
\newcommand*{\theorem}[2]{\textbf{Теорема #1. } #2 \newline}
\newcommand*{\notabene}[1]{\textit{Notabene. #1.} \newline}
\newcommand*{\definition}[1]{\textbf{Определение.} #1 \newline}
\newcommand*{\prooff}[1]{$\vartriangleright$ #1 $\blacktriangleleft$}
\newcommand*{\R}{\mathbb{R}}
\newcommand*{\eps}{\varepsilon}
\newcommand*{\cf}[2]{\cfrac{#1}{#2}}
\newcommand*{\D}{\Delta}
\newcommand*{\p}[1][n]{P_{#1}(x)}
\newcommand*{\Q}[1][m]{Q_{#1}(x)}
\newcommand*{\Rfrac}[2]{\frac{\p{#1}}{\Q{#2}}}
\newcommand*{\sfrac}{\frac{A}{(x - a)^k}}
\newcommand*{\Sfrac}{\frac{Ax+B}{(x^2 + px + q)^k}}
\newcommand*{\PAdv}[2]{P_{#1}^{#2}(x)}
\newcommand*{\QAdv}[2]{Q_{#1}^{#2}(x)}
\newcommand*{\rome}[1]{\MakeUppercase{\MakeUppercase{\romannumeral #1}}}
\begin{document}

\maketitle
\newpage
\tableofcontents
\newpage

\section{Несобственный интеграл}
\definition{Пусть $E \subset \mathbb{R}, f(x) \in R_{loc}(E)$ - локально интегрируема на $E$, если $f \in R[a, b]$ для $\forall[a, b] \subset E$\\ 
}
\definition{Пусть $f \in R_{loc}[a, b)$, где $(b \in \mathbb{R} \cup {+\infty})$ 
}
\begin{align*}
    &F(\omega) = \int_{a}^{\omega} f(x)dx, \text{ при }\omega \in [a, b)\\
    &\int_{a}^{b} f(x)dx = \displaystyle \lim_{\omega \to b - 0} F(\omega) \text{ - несобственный интеграл}
\end{align*}
Если $\displaystyle \lim F(\omega)$ конечен, то $\int_{a}^{b} f(x)dx$ сходится
\begin{itemize}
    \item Если $\int_{a}^{b}$в $\R \cup \{+\infty\}$ - $\rome{1}$ рода
    \item Если $\int_{a}^{b}$в $\R$ - $\rome{2}$ рода
\end{itemize}
Если $f \notin R[a, \bar{b}]$, то $\bar{b}$ называется \textit{особой точкой $f(x)$}.
Пример:
\begin{align*}
    &\int_{0}^{+\infty}\frac{\sin x}{x}dx \text\ -\ \rome{1} \text{ рода ($+\infty$ особая точка)}\\
    &\int_{0}^{1}\frac{dx}{x} - \rome{2} \text{ рода (0 особая точка)}
\end{align*}
Если:
\begin{equation*}
    \int_{a}^{\bar{b}}f(x)dx = \displaystyle \lim_{\omega \to b-0} \int_{a}^{\omega} \Rightarrow
    \begin{cases}
        \in \R, \text{то сходится},
        \\
        \notin \R, \text{то расходится} 
        \begin{cases}
            \pm\infty 
            \\
            \nexists\ \overline{\R}
        \end{cases}
    \end{cases}
\end{equation*}
Например:
\begin{equation*}
    \int_{1}^{+\infty}\frac{dx}{x^{\alpha}} =
    \begin{cases}
        \ln x \biggm\vert_{1}^{+\infty},\ \alpha = 1
        \\
        \frac{x^{1 - \alpha}}{1 - \alpha}\biggm\vert_{1}^{+\infty}, \alpha \neq 1
    \end{cases}
    =
    \begin{cases}
        +\infty,\ \alpha = 1
        \\
        +\infty,\ \alpha < 1
        \\
        \frac{1}{\alpha - 1},\ \alpha > 1
    \end{cases}
\end{equation*}
\begin{equation*}
    \int_{0}^{1} \frac{dx}{x^{\alpha}} \text{ сходится при $\alpha < 1$. Расходится при $\alpha \geq 1$}
\end{equation*}
\subsection{Свойства несобственного интеграла}
\begin{enumerate}
    \item {
        Линейность. $f, g \in R_{loc}{[a, b)}$
        \begin{itemize}
            \item $$\int_{a}^{b} (f + g)dx = \int_{a}^{b}f(x)dx + \int_{a}^{b}g(x)dx$$если $$\int_{a}^{b}f(x)dx, \int_{a}^{b}g(x)dx \in \Bar{\mathbb{R}}$$
            \item $$\int_{a}^{b} \alpha \cdot f(x)dx = \alpha \cdot \int_{a}^{b} f(x)dx$$
            \item {
                \textbf{Следствие:}\\
                Сходится + Сходится = Сходится\\
                Сходится + Расходится = Расходится\\
                Расходится + Расходится = ?
            }
        \end{itemize}
    }
    \item {
        Монотонность. $f, g \in R_{loc}[a, b), f(x) \leq g(x) \text{ на }[a, b) \text{ и }\int_{a}^{b}f(x)dx, \int_{a}^{b}g(x)dx \in \overline{R}: $\\
        Тогда: $$\int_{a}^{b}f(x)dx \leq \int_{a}^{b}g(x)dx$$   
    }
    \item {
        Аддитивность. $f \in R_{loc}[a, b), c \in (a, b):$\\
        Тогда: $$\int_{a}^{b}f(x)dx = \int_{a}^{c}f(x)dx + \int_{c}^{b}f(x)dx$$
        \prooff{
            $$
                \int_{a}^{b}f(x)dx = \displaystyle \lim_{\omega \to b-0} \int_{a}^{c} + \int_{c}^{\omega} = \int_{a}^{c} + \displaystyle \lim_{\omega \to b-0} \int_{c}^{\omega}
            $$
        }
    }
    \item {
        Формула по частям. $u, v - \text{ дифференцируема на } [a, b), u^{'} v^{'} \in R_{loc}[a, b)$
        $$
            \int_{a}^{b}u(x)\cdot v^{'}(x) = u v \biggm \vert_{a}^{b} - \int_{a}^{b}v(x)\cdot u^{'}(x)dx
        $$
    }
    \item {
        Формула замены переменной. $f \in C[a, b), x = \phi(t)\ [\alpha, \beta) \to [a, b).\ \phi - \text{ диффернцируема на }[\alpha, \beta),\ \\ \exists \phi(\beta - 0) = \displaystyle \lim_{t \to \beta - 0} \phi(t) \text{ в }\overline{R}$\\
        Тогда:
        $$
            \int_{\phi(\beta-0)}^{\phi(\alpha)}f(x)dx = \int_{\alpha}^{\beta}f(\phi(t))\cdot \phi^{'}(t)dt
        $$
        и эти интегралы (не)существуют одновременно\\
        \prooff{
            \begin{enumerate}
                \item {
                    $\exists I_1 \in \overline{R},\\ I_2 = \displaystyle \lim_{\omega \to \beta - 0}\int_{\alpha}^{\omega} f(\phi)\cdot \phi^{'} dt = \displaystyle \lim_{\omega \to \beta - 0} \int_{\phi(\alpha)}^{\phi(\omega)}f(x)dx = \int_{\phi(\alpha)}^{\phi(\beta - 0)}f(x)dx = I_1$
                }
                \item {
                    $\exists I_2 \in \overline{R},\\ I_1 = \displaystyle \lim_{\omega \to \phi(\beta - 0)} \int_{\phi(\alpha)}^{\phi(\beta - 0)}f(x)dx$
                    \begin{itemize}
                        \item {
                            Если $\phi(\beta - 0) < b$, то очевидно
                        }
                        \item {
                            Пусть $x_n \to \phi(\beta - 0) = b, x_n \in [\phi(\alpha), b)$\\
                            для $\forall n\  \exists \gamma_n: \phi(\gamma_n) = x_n$\\
                            Докажем, что $$\gamma_n \to \beta - 0,\ \text{при } n \to +\infty$$
                            Если $\gamma \nrightarrow \beta - 0,$ то $\exists \varepsilon > 0\ \forall n_0:\  \exists n \geq n_0$
                            \begin{align*}
                                &\gamma_n < \beta - \varepsilon\\
                                &\phi(\gamma_n) \leq \sup_{[\alpha, \beta - \varepsilon]}\phi < b
                            \end{align*}
                            Отсюда следует, что $\phi(\gamma_n) \nrightarrow b$
                        }
                    \end{itemize}
                    Итак, для $\forall x_n \to \phi(\beta - 0):\ $
                    $$
                        \int_{\phi(\alpha)}^{x_n} f(x)dx = \int_{\alpha}^{\gamma_n} f(\phi(t))\cdot\phi^{'}(t)dt \xrightarrow[n \to \infty]{} \int_{\alpha}^{\beta - 0} f(\phi)\cdot\phi^{'}(t)dt
                    $$
                }
            \end{enumerate}
        } 
    }
\end{enumerate}
\subsection{Интеграл с несколькими особыми точками}
1.
$$
    f \in R(a, b)
    \int_{a}^{b} f(x)dx = \displaystyle \lim_{\omega \to a + 0} \int_{\omega}^{c} f(x)dx + \displaystyle \lim_{\omega \to b - 0} \int_{c}^{\omega} f(x)dx = \int_{a}^{c} f(x)dx + \int_{c}^{b} f(x)dx
$$
где $c \in (a, b)$
$$
    \int_{a}^{b} f(x)dx \text{ сходится} \Leftrightarrow \int_{a}^{c}f(x)dx \text{ и } \int_{c}^{b} f(x)dx \text{ сходятся}
$$
2.
Пусть $f$ определена на $(a, b); a, b \in \bar{\R}$ кроме конечного числа точек. $C \in (a, b)$ называется особой точкой $f$, если $f \notin R[\alpha, \beta]$ для $\forall \alpha, \beta: a < \alpha < \beta < b$\\
$a (b)$ - особая, если $a = -\infty (b = -\infty)$ или $f \notin R[a, \alpha] \forall \alpha \in (a, b)$\\\\
Пусть $c_1 < c_2 < ... < c_{n-1}$ - особые точки $f$ на $(a, b)$ \\
$c_0 = a, c_n = b$ 
\begin{align*}
    \int_{a}^{b} &f(x)dx = \sum_{i = 1}^{n} \int_{c_{i - 1}}^{c_i} f(x)dx \\
    \int_{a}^{b} &f(x)dx \text{ - сходится} \Leftrightarrow \int_{c_{i - 1}}^{c_i} f(x)dx \text{ - сходится } \forall i = 1...n \\
\end{align*}
Проблема: если пытаться взять существенно большие значения для $b$, то зачастую численный метод вычисления несобственного интеграла даст сильно неточное значение.\\ Например:
\begin{align*}
    &\int_{1}^{+\infty} \frac{dx}{x} = +\infty\\
    &\int_{1}^{10^6} \frac{dx}{x} = 6 \cdot \ln10 \approx 13\\
    &\int_{1}^{10^{10}} \frac{dx}{x} = 10 \cdot \ln 10 \approx 23
\end{align*}
\subsection{Интеграл от знакопостоянной функции}
Пусть $f \in R_{loc}[a, b), f \geq 0 $ на $[a, b)$
\theorem{}{Пусть $F(\omega) = \int_{a}^{\omega} f(x)dx$
    Тогда:
    1. $F \uparrow$
    2. $\int_{a}^{b} f(x)dx$ - сходится $\Leftrightarrow F$ ограничена 
}
\textit{Доказательство: \\}
1.
$$
    \omega_1 < \omega_2,\  f(\omega_2) = \int_{a}^{\omega_2} = \int_{a}^{\omega_1} + \int_{\omega_1}^{\omega_2} \geq F(\omega_1)
$$
2. Очевидно \\\\
\theorem{(признаки сравнения)} {}
Пусть $f, q \in R_{loc}[a, b)$
\begin{enumerate}
    \item {$0 \leq f(x) \leq g(x) $ на $[a, b)$. Тогда
        \begin{itemize}
            \item Если $\int_{a}^{b} g(x)dx$ - сходится $\Rightarrow \int_{a}^{b} f(x)dx$ - сходится
            \item Если $\int_{a}^{b} g(x)dx$ - расходится $\Rightarrow \int_{a}^{b} f(x)dx$ - расходится
        \end{itemize}
    }
    \item {$f \thicksim g$, при $x \to b - 0$. Тогда
        \begin{itemize}
            \item $\int_{a}^{b} f(x)dx$ и $\int_{a}^{b} g(x)dx$ оба сходятся или оба расходятся 
        \end{itemize}
    }
\end{enumerate}
\textit{Доказательство: \\}
1. Очевидно \\
2. Пусть $f(x) \thicksim g(x)$ при $x \to b - 0$
\begin{align*}
    &f(x) = \alpha(x) \cdot g(x), \alpha(x)\xrightarrow[x \to b - 0]{} 1\\
    &\frac{1}{2} \leq \alpha(x) \leq \frac{3}{2} \text{ для } x \in (\delta, b)\\
    &\frac{1}{2} g(x) \leq f(x) \leq \frac{3}{2} g(x)
\end{align*}
\subsection{Интеграл от знакопеременной функции}
\theorem{(Критерий Коши)}{
    \begin{align*}
        \text{Пусть }f \in R_{loc}[a, b). &\text{Тогда }\int_{a}^{\bar{b}} f(x)dx \text{ сходится } \Leftrightarrow\\
        \forall \epsilon > 0 \exists \Delta \in (a, b):\  &\forall \delta_1, \delta_2 \in (\Delta, b) \Rightarrow\\
        &\left|\int_{\delta_1}^{\delta_2} f(x)dx\right| < \epsilon
    \end{align*}
}
\textit{Доказательство: \\}
\begin{align*}
    &F(\omega) = \int_{a}^{\omega} f(x)dx. \exists \displaystyle \lim_{\omega \to b - 0} F(\omega) \in \R \Leftrightarrow\\
    &\forall \varepsilon > 0\ \exists \Delta:\ \forall \delta_1, \delta_2 > \Delta\ |F(\delta_1) - F(\delta_2)| < \varepsilon
\end{align*}
\definition{Пусть $f \in R_{loc}[a, b)$
    $$
        \int_{a}^{b} f(x)dx \text{ сходится абсолютно, если } \int_{a}^{b} |f(x)|dx \text{ сходится}
    $$
}
\theorem{}{}
Если $\int_{a}^{b} f(x)dx$ сходится абсолютно, то он сходится\\\\
\textit{Доказательство: }
$$
    \int_{a}^{b}|f(x)|dx \text{ сходится } \Rightarrow \forall \varepsilon > 0\  \exists \Delta \in (a, b)\ \forall \delta_1, \delta_2 > \Delta: \left|\int_{\delta_1}^{\delta_2}f(x)dx\right| \leq \left|\int_{\delta_1}^{\delta_2} f(x)dx\right| < \varepsilon
$$
\theorem{(о сумме с абсолютно сходящимся интегралом)}{}
Пусть $f, g, h \in R_{loc}[a, b), f(x) = g(x) + h(x)$ на $[a, b)$ и $\int_{a}^{b}h(x)dx$ сходится абсолютно.\\ 
Тогда:
\begin{itemize}
    \item $\int_{a}^{b}f(x)dx$ и $\int_{a}^{b}g(x)dx$ - оба сходятся абсолютно\\
    \item $\int_{a}^{b}f(x)dx$ и $\int_{a}^{b}g(x)dx$ - оба сходятся условно\\
    \item $\int_{a}^{b}f(x)dx$ и $\int_{a}^{b}g(x)dx$ - оба расходятся.\\\\
\end{itemize}
\textit{Доказательство: \\}
$$
    |f| \leq |g| + |h|
$$
Пример: 
$$
    \int_{1}^{+\infty}\frac{\sin x}{x}dx = -\frac{\cos x}{x}\biggm\vert_{1}^{+\infty} - \int_{1}^{+\infty} \frac{\cos x}{x^2} dx
$$
Рассмотрим: 
$$ 
    \int_{1}^{+\infty} \frac{|\sin x|}{x}dx
$$
$$
    \delta_1 = \pi n, \delta_2 = 2\pi n
$$
$$
    \int_{\pi n}^{2 \pi n} \frac{|\sin x|}{x}dx \geq \frac{1}{2 \pi n} \int_{\pi n}^{2 \pi n}|\sin x|dx =
$$
$$
    \frac{1}{2 \pi n} \cdot n \cdot \int_{0}^{\pi} \sin x dx = \frac{1}{\pi} = \varepsilon
$$
\Large{\textit{Вопросы из чата:\\\\}}
Пусть $f \in R_{loc}[a, +\infty)$. Правда ли, что: 
$$
    \int_{a}^{+\infty}f(x)dx \text{ - сходится }\Rightarrow \displaystyle \lim_{x \to +\infty}f(x) = 0
$$
Ответ: нет, неправда.
$$
    f \in C[a, +\infty),\ f \geq 0 \nRightarrow f(x) \text{ - ограничена}
$$
\theorem{(Признак Абеля-Дирихле)}{}\\
Пусть $f(x) \in C[a, b),\ g(x) \in C^1[a, b)$. 
Тогда для сходимости $\int_{a}^{b} f(x) \cdot g(x) dx$ достаточно одной пары условий
\begin{itemize}
    \item {
        Признак Дирихле
        \begin{enumerate}
            \item $F(\omega) = \int_{a}^{\omega}g(x)dx \text{ - ограничена}$
            \item $g(x) \text{ монотонна и }g(x) \xrightarrow[x \to b - 0]{}0$
        \end{enumerate}
    }
    \item {
        Признак Абеля: 
        \begin{enumerate}
            \item $\int_{a}^{b} f(x)dx \text{ сходится }$
            \item $g(x) \text { монотонна и ограничена}$
        \end{enumerate}
    }
\end{itemize}
\prooff{
    \begin{enumerate}
        \item Дирихле: {
            $F^{'}(x) = f(x)$\\\\
            Критерий Коши: $\int_{a}^{b}f(x) \cdot g(x)dx \text{ - сходится }\Leftrightarrow \forall \varepsilon > 0\ \exists \Delta \in (a, b)\ \forall\delta_1, \delta_2 \in (\Delta, b)$
            $$
                \left|\int_{\delta_1}^{\delta_2}f(x)\cdot g(x)dx\right| = \left|\int_{\delta_1}^{\delta_2}f(x)d(F)\right| = \left|g(x) \cdot F(x)\biggm\vert_{\delta_1}^{\delta_2} = \int_{\delta_1}^{\delta_2}F(x) \cdot g^{'}(x)dx\right|\leq
            $$
            $$
                \leq C \cdot (|g(\delta_1)|+|g(\delta_2)|) + C \cdot \left|\int_{\delta_1}^{\delta_2}|g^{'}(x)dx|\right| \leq 2C(|\delta_1|+|g(\delta_2|) \leq 4C \cdot \varepsilon 
            $$
        }
        \item Абель:{
            т.к $g(x)$ - монотонна и ограничена $\Rightarrow$
            $$
                \exists \displaystyle \lim_{x \to b - 0} g(x) = A \in \R
            $$
            $$
                g(x) - A \text{ - монотонна и } \xrightarrow[x \to b - 0]{}0
            $$
            Рассмотрим
            $$
                \int_{a}^{b} f(x) \cdot g(x) dx = \int_{a}^{b} f(x) \cdot (g - A)dx + A \cdot \int_{a}^{b} f(x)dx \Rightarrow \text{ сходится по Дирихле}
            $$  
        }
    \end{enumerate}
}
\\
Пример: 
$$
    \int_{0}^{+\infty} \frac{\sin(x)}{x}dx
$$
$f(x) = \sin(x);\\\\ F(\omega) = \int_{0}^{\omega}\sin(x)dx = 1 - \cos(\omega)$ - ограничена;\\\\ $g(x) = \frac{1}{x} \xrightarrow[x \to + \infty]{}0$ и $\downarrow\\\\$
Отсюда следует, что: 
$$
    \int_{0}^{+\infty} \frac{\sin(x)}{x}dx \text{ - сходится}
$$
\subsection{Интеграл в смысле главного значения}
$$
    \int_{-\infty}^{+\infty} = \int_{-\infty}^{c} + \int_{c}^{+\infty} = \displaystyle \lim_{a \to -\infty} \int_{a}^{c} + \lim_{b \to +\infty} \int_{c}^{b}
$$
Из того, что мы знаем на данный момент, мы делаем вывод, что интеграл расходится. Чтобы учитывать подобные ситуации, вводят понятние:\\\\
\definition{ Пусть $f \in R_{loc}(\R)$. Тогда $v.p \int_{-\infty}^{+\infty} f(x)dx = \displaystyle \lim_{a \to +\infty} \int_{-a}^{a}f(x)dx$}\\
\definition{ Пусть $f \in R_{loc}([a, b] \backslash \{c\}) \Rightarrow v.p \int_{a}^{b} = \lim_{\varepsilon \to 0+}\left(\int_{a}^{c - \varepsilon}f(x)dx + \int_{c + \varepsilon}^{b}f(x)dx\right)$}
\lemma{
    Если сходится $\int_{a}^{b}$, то сходится и $v.p \int_{a}^{b} = \int_{a}^{b}$ 
}
\newpage
\subsection{\textit{Полезные вещи $:\bigl)$}} 
\begin{enumerate}
    \item Интеграл Эйлера-Пуассона:\\ {
        $$
            \int_{-\infty}^{+\infty} e^{-x^2}dx = \sqrt{\pi}
        $$
        \prooff{
            $$
            e^x \geq x + 1,\ \forall x \in \R
            $$
            $$
                1 - x^2 \leq e^{-x^2} = \frac{1}{e^{x^2}} \leq \frac{1}{x^2 + 1}
            $$
            $$
                (1 - x^2)^k \leq e^{-kx^2} \leq \frac{1}{(x^2 + 1)^k},\ k \in \mathbb{N}
            $$
            $$
                \int_{-1}^{1}(1 - x^2)^k dx \leq \int_{-1}^{1}e^{-kx^2}dx \leq \int_{-\infty}^{+\infty}e^{-kx^2}dx \leq \int_{-\infty}^{+\infty} \frac{dx}{(1 + x^2)^k}
            $$
            В левой части делаем замену $x = \sin(t)$, в правой части сделаем замену $\tan(t)$
            $$
                \int_{-\frac{\pi}{2}}^{\frac{\pi}{2}}\cos^{2k+1}(t)dt \leq \int_{-\infty}^{+\infty}e^{-kx^2}dx \leq \int_{-\frac{\pi}{2}}^{\frac{\pi}{2}}\cos^{2k-2}(t)dt
            $$
            $$
                \int_{0}^{\frac{\pi}{2}} \cos^n(x)dx = \int_{0}^{\frac{\pi}{2}} \sin^n(x) = ?
            $$
            $$
                2 \cdot \frac{(2k)!!}{(2k+1)!!} \leq \int_{-\infty}^{+\infty}e^{-kx^2} dx \leq 2 \cdot \frac{\pi}{2} \cdot \frac{(2k - 3)!!}{(2k - 2)!!}
            $$
            Делаем замену $t = \sqrt{k} \cdot x$
            $$
                \frac{2\sqrt{k}}{(2k + 1)} \cdot \frac{1}{\sqrt{k}}\cdot \frac{(2k)!!}{(2k - 1)!!} \leq \frac{1}{\sqrt{k}} \cdot \int_{-\infty}^{+\infty}e^{-t^2}dt \leq \pi \cdot \frac{2k \cdot \sqrt{k}}{(2k - 1)} \cdot \frac{(2k - 1)!!}{(2k)!!} \Rightarrow
            $$
            $$
                \int_{-\infty}^{+\infty}e^{-t^2}dt = \sqrt{\pi}
            $$
        }
    } 
\end{enumerate}
\end{document}