\documentclass{article}
\usepackage[utf8]{inputenc}
\usepackage[T2A]{fontenc}
\usepackage[russian]{babel}
\usepackage[normalem]{ulem}
\usepackage{amsfonts}
\usepackage{amsmath}
\usepackage{amsthm}
\usepackage{amssymb}
\usepackage{arcs}
\usepackage{fancyhdr}
\usepackage{float}
\usepackage[left=2cm,right=2cm,top=2cm,bottom=2cm]{geometry}
\usepackage{graphicx}
\usepackage{hyperref}
\usepackage{multicol}
\usepackage{stackrel}
\usepackage{xcolor}
\usepackage{cancel}

\makeatother
\makeatletter

\title{\textbf{Билеты к коллоку}}
\author{i.g. i.a.}
\date{x марта 2023 г.}

% \newcommand*{\limToInf}[2]{\displaystyle \lim_{#1 \to \infty} #2}
% \newcommand*{\limToZero}[2]{\displaystyle \lim_{#1 \to 0} #2}
\newcommand*{\lemma}[1]{\textbf{Лемма.} #1. \newline}
\newcommand*{\theorem}[2]{\textbf{Теорема#1. } #2 \newline}
\newcommand*{\notabene}[1]{\textit{Notabene. #1.} \newline}
\newcommand*{\definition}[1]{\textbf{Определение.} #1 \newline}
\newcommand*{\Hom}[2]{Hom$_\mathbb{K}(#1, #2)$}
% \newcommand*{\eps}{\varepsilon}
% \newcommand*{\cf}[2]{\cfrac{#1}{#2}}
% \newcommand*{\D}{\Delta}
% \newcommand*{\p}[1][n]{P_{#1}(x)}
% \newcommand*{\Q}[1][m]{Q_{#1}(x)}
% \newcommand*{\Rfrac}[2]{\frac{\p{#1}}{\Q{#2}}}
% \newcommand*{\sfrac}{\frac{A}{(x - a)^k}}
% \newcommand*{\Sfrac}{\frac{Ax+B}{(x^2 + px + q)^k}}
% \newcommand*{\PAdv}[2]{P_{#1}^{#2}(x)}
% \newcommand*{\QAdv}[2]{Q_{#1}^{#2}(x)}

\begin{document}
\tableofcontents
\maketitle

\section{Линейный оператора: определение, примеры}
\definition{Отображение $\varphi: X \to Y$ линейного пространства $X$ в линейное пространство $Y$ называется \textbf{линейным оператором}, если $\forall x, x_1, x_2 \in X, \quad \forall \alpha \in \mathbb{K}$}
$$
    \varphi(x_1 + x_2) = \varphi(x_1) + \varphi(x_2), \quad \varphi(\alpha x) = \alpha \varphi(x)
$$
\notabene{Множество линейных операторов из $X(\mathbb{K})$ в $Y(\mathbb{K})$ обозначается \Hom{X}{Y}}
\newline
\notabene{Оператор $\varphi: X \to X$, отображающий $X$ в себя, называют \textit{эндоморфизмом} и пишут $\varphi \in \text{End}(X)$, а в случае отображения на себя - \textit{автоморфизмом}}
\newline
\textbf{Примеры:} 
\begin{enumerate}
    \item \textit{Нульоператор: } $\Theta: X \to Y, \quad \Theta x = 0_Y$
    \item \textit{Единичный оператор или тождественный: } $\mathcal I: X \to Y, \quad \mathcal Ix = x$
    \item \textit{Проекторы: } $\mathcal P_{L_1}^{|| L_2}: X \to X, \quad X = L_1 \oplus L_2 \quad \mathcal P_{L_1}^{|| L_2} x = x_1, \quad x_1 \in L_1$
\end{enumerate}
\section{Ядро и образ линейного оператора: теорема о ядре и образе}
\definition{\textbf{Ядром} линейного оператора $\varphi \in$ \Hom{X}{Y} называется подмножество $X$: }
$$
    \ker \varphi = \{ x \in X: \quad \varphi(x) = 0\}
$$
\definition{\textbf{Образом} линейного отображения $\varphi \in$ \Hom{X}{Y} называется подмножество $Y$:}
$$
    \text{Im} \varphi = \{ y \in Y: \quad \exists x \in X \quad \varphi(x) = y\} = \varphi(X)
$$
\notabene{Образ и ядро являются линейными подпространствами.}
\theorem{(база)}{Пусть $\varphi \in$ \Hom{X}{Y}, тогда имеет место изоморфизм}
$$
    X/\ker \varphi \simeq \text{Im} \varphi
$$
\begin{proof}
    Отображение $\bar \varphi: X/\ker \varphi \to \text{Im} \varphi$, заданное как 
    $$
        x + \ker \varphi \mapsto \varphi(x)
    $$
    Гомоморфно (выполнена линейность), сюрьективно и инъективно - а значит является изоморфизмом.
\end{proof}
.\newline
\definition{Пусть $L \leq X$ - линейное подпространство $X$. Набор $\{ v_j \}^m_{j = 1} \in X$ называется \textbf{линейно независимым над $\mathbb{K}$ относительно $L$}, если}
$$
    \lambda_1 v_1 + \lambda_2 v_2 + \dots + \lambda_m v_m \in L \quad \Rightarrow \quad \lambda_1 = \lambda_2 = \dots = \lambda_m = 0
$$
то есть набор векторов ЛНЗ между собой и любыми векторами из $L$.
\newline
\newline
\definition{Говорят, что $\{ v_j \}^m_{j = 1}$ \textbf{порождает} $X$ относительно $L$, если}
$$
    X = \langle v_1, v_2, v_3, \dots, v_m \rangle + L
$$
то есть $X$ это сумма линейной оболочки набора и подпространства $L$.
\newline 
\newline 
\definition{Говорят, что $\{ v_j \}^m_{j = 1}$ базис $X$ относительно $L$, если набор ЛНЗ над $L$ и порождает $X$ относительно $L$}
\lemma{Следующие условия эквивалентны: }
\begin{enumerate}
    \item $\{ v_j \}^m_{j = 1}$ - базис $X$ относительно $L$;
    \item $\{ \bar v_1, \bar v_2, \bar v_3, \dots, \bar v_m \}$ - базис $X/L$;
    \item $X = \{ v_1, v_2, v_3, \dots, v_m\} \oplus L$.
\end{enumerate}
\begin{proof}
    Очевидно, но поясню
    \begin{enumerate}
        \item $1 \Rightarrow 2$: 
        Так как набор - базис, значит ЛНЗ в $X$ относительно $L$, из чего следует ЛНЗ в $X/L$, а полнота следует отсюда же. 
        \item $2 \Rightarrow 3$: 
        Очевидно
        \item $3 \Rightarrow 1$: 
        Очевидно
    \end{enumerate}
\end{proof}
\lemma{Если $L \leq X$, тогда имеет место:}
$$
    \dim X = \dim L + \dim X/L
$$
доказательство очевидно следует из прошлой леммы и к тому же, так как линейная оболочка ЛНЗ набора в пересечении с $L$ дает пустое множество - то сумма является прямой и из нее следует сумма размерностей.
\newline 
\newline 
\theorem{(о ядре и образе)}{Пусть $\varphi \in$ \Hom{X}{Y}, тогда имеет место}
$$
    \dim \ker \varphi + \dim \text{Im}\varphi = \dim X
$$
\begin{proof}
    Так как $\text{Im} \varphi \simeq X/\ker \varphi$, то $\dim \text{Im}\varphi = \dim X/\ker\varphi$, тогда из прошлой леммы получим требуемое.
\end{proof}
\section{Пространство линейных операторов}
\definition{Говорят, что операторы $\varphi, \psi \in$ \Hom X Y \textbf{равны}, если}
$$
    \varphi x = \psi x, \quad \forall x \in X
$$
\definition{Отображение $\chi$ называется \textbf{суммой} линейных операторов $\varphi, \psi \in$ \Hom X Y , если}
$$
    \forall x \in X \quad \chi(x) = \varphi(x) + \psi(x)
$$
\lemma{Имеет место $\chi \in$ \Hom X Y}
$$
    \chi(x_1 + x_2) = \chi(x_1) + \chi(x_2)
$$
$$
    \chi(\lambda x) = \lambda \chi(x)
$$
\begin{proof}
    $$
        \chi(x_1 + x_2) = \varphi(x_1 + x_2) + \psi(x_1 + x_2) = (\varphi + \psi)x_1 + (\varphi + \psi)x_2 = \chi(x_1) + \chi(x_2)
    $$
    $$
        \chi(\lambda x) = (\varphi + \psi)\lambda x = \lambda(\varphi + \psi)x = \lambda \chi (x)
    $$
\end{proof}
.\newline 
\definition{Отображение $\zeta$ называется \textbf{умножением} линейного оператора $\varphi$ на число $\lambda$, если}
$$
    \forall x \in X \quad \zeta x = \lambda \varphi(x)
$$
\lemma{Имеет место $\zeta \in$ \Hom X Y}
\begin{proof}
    Очевидно. Так же, как для суммы.
    $$
        \zeta(x_1 + x_2) = \lambda (\varphi(x_1) + \varphi(x_2)) = \lambda \varphi(x_1) + \lambda \varphi(x_2) = \zeta(x_1) + \zeta(x_2) 
    $$
    $$
        \zeta(\lambda_1 x) = \lambda_1\lambda_2 \varphi(x) = \lambda_1 \zeta(x)
    $$
\end{proof}
.\newline
\theorem{}{Множество \Hom X Y - линейное пространство над полем $\mathbb{K}$.}
\begin{proof}
    Пусть $\varphi, \psi, \zeta \in$ \Hom X Y, а $\lambda, \mu \in \mathbb{K}$.
    \begin{enumerate}
        \item Коммутативность сложения
        $$
            \varphi + \psi = \psi + \varphi 
        $$ 
        \item Ассоциативность сложения 
        $$
            \varphi + (\psi + \zeta) = (\varphi + \psi) + \zeta
        $$
        \item Существование нуля
        $$
            \exists \varTheta: \quad \varphi(x) + \varTheta(x) = \varphi(x) + 0_Y = \varphi(x)
        $$
        \item Существование противоположного элемента
        $$
            -\varphi + \varphi = \varTheta
        $$
        \item 
        $$
            \lambda(\varphi + \psi) = \lambda\varphi + \lambda\psi
        $$
        \item 
        $$
            (\lambda + \mu) \varphi = \lambda\varphi + \mu \varphi
        $$
        \item 
        $$
            \lambda(\mu\varphi) = (\lambda\mu)\varphi
        $$
        \item 
        $$
            \exists \mathbb I: \quad \mathbb I \cdot \varphi = \varphi
        $$
    \end{enumerate}
\end{proof}
\section{Матрица линейного оператора}
\definition{\textbf{Матрицей линейного оператора} $\varphi$ в паре базисов $\{ e_i\}^n_{i = 1}$ и $\{ g_j \}^m_{j = 1}$ называется матрица $A$, по столбцам которой координаты образов векторов базиса $\{ e_i \}^n_{i = 1}$ в базисе $\{ g_j\}^m_{j = 1}$}
$$
    \varphi(e_i) = \sum_{j = 1}^{m} \alpha^j_ig_j
$$
\theorem{}{Задание линейного оператора $\varphi$ эквивалентно заданию его матрицы в фиксированной паре базисов.}
\begin{proof}
    Докажем импликации
    \begin{itemize}
        \item[$\Rightarrow$] Очевидно. То есть задали оператор - автоматически можем задать матрицу по определению.
        \item[$\Leftarrow$] Пусть $\varphi \in$ \Hom X Y - линейный оператор и $\{ e_i\}^n_{i = 1}, \{ g_j \}^m_{j = 1}$ - базисы пространств $X, Y$ соответственно. Рассмотрим элементы $x \in X, y \in Y$ такие, что
        $$
            x = \sum_{i = 1}^{n} \xi^i e_i, \quad y = \sum_{j = 1}^{m} \eta^j g_j, \quad \varphi(x) = y
        $$
        Рассмотрим действие оператора: 
        $$
            \varphi(x) = \varphi(\sum_{i = 1}^{n} \xi^i e_i) = \sum_{i = 1}^{n} \xi^i \varphi(e_i) = \sum_{i = 1}^{n} \xi^i \sum_{j = 1}^{m} \alpha^j_i g_j = \sum_{j = 1}^{m} \eta^j g_j
        $$
        Откуда следует, мелкими преобразованиями
        $$
            \eta^j = \sum_{i = 1}^{n} \xi^i \alpha^j_i
        $$  
        то есть мы смогли получить действие оператора на вектор, зная коэффициенты матрицы линейного оператора.
    \end{itemize}
\end{proof} 
\section{Теорема о базисе пространства линейных операторов}
\theorem{}{Набор операторов $\{ ^i_j\varepsilon \}$, действующих на произвольный вектор $x \in X$ по правилу}
$$
    ^i_j \varepsilon(x) = \xi^ig_j, \quad x = \sum_{i = 1}^{n} \xi^i e_i
$$
образует базис пространства \Hom X Y.
\begin{proof}
    Необходимо показать, что набор операторов $\{ ^i_k \varepsilon \}$ является полным и ЛНЗ в \Hom X Y: 
    \begin{itemize}
        \item[ПН:] Пусть $\varphi \in$ \Hom X Y, тогда
        $$
            \varphi(x) = \varphi(\sum_{i = 1}^{n} \xi^i e_i) = \sum_{i = 1}^{n}\xi^i \varphi(e_i) = \sum_{i = 1}^{n} \xi^i \sum_{j = 1}^{m} a^j_i g_j =
        $$
        $$
            = \sum_{i = 1}^{n} \sum_{j = 1}^{m} \xi^i a_i^j g_j = \sum_{i = 1}^{n} \sum_{j = 1}^{m} .^i_j \varepsilon(x)a^j_i, \quad \forall x \in X
        $$
        Откуда следует, что 
        $$
            \varphi = \sum_{i = 1}^{n} \sum_{j = 1}^{m} .^i_j \varepsilon a^j_i, \quad \forall \varphi \in \text{Hom}(X, Y)
        $$
        \item[ЛНЗ:] рассмотрим линейную комбинацию векторов набора $\{ ^i_j \varepsilon \}$:
        $$
            \sum_{i = 1}^{n} \sum_{j = 1}^{m} .^i_j \varepsilon \beta^j_i = \varTheta
        $$
        и применим обе части равенства к базисному элементу $e_k$: 
        $$
            \sum_{i = 1}^{n} \sum_{j = 1}^{m} .^i_j \varepsilon(e_k)\beta^j_i = \sum_{i = 1}^{n} \sum_{j = 1}^{m} \delta ^i_k g_j \beta^j_i = \sum_{j = 1}^{m} g_j \beta^j_k = 0_Y
        $$
        Но так как набор $\{ g_j \}^m_{j = 1}$ - ЛНЗ, а значит $\beta^j_k = 0 \forall k$. А так как $k$ - любое, то это верно для всех векторов набора $\{e_i\}^n_{i = 1}$. 
    \end{itemize}
\end{proof}
.\newline 
\subsection{То, чего нет в билетах, но оно важно}
\definition{ \textbf{Преобразованием подобия} матрицы $A$ называется преобразование}
$$
    A \mapsto T^{-1} \cdot A \cdot T, \quad \det T \neq 0
$$
что позволяет находить матрицу оператора для другого базиса, зная старый и новый базисы и матрицу ЛО.
\section{Композиция линейных операторов}
\definition{Пусть $X, Y, K$ - линейные пространства. \textbf{Композицией} линейных операторов $\psi \in$ \Hom Y Z и $\varphi \in$ \Hom X Y называется отображение $\chi \in$ \Hom X Z такое, что}
$$
    \chi = \psi \circ \varphi \quad \psi \varphi x = \psi(\varphi x) \quad \forall x \in X
$$
\lemma{Отображение $\chi$ - линейный оператор}
\begin{proof}
    $$
        \chi (x_1 + x_2) = \psi(\varphi(x_1 + x_2)) = \psi(\varphi x_1 + \varphi x_2) = \psi (\varphi x_1) + \psi(\varphi x_2) = \chi x_1 + \chi x_2
    $$
    $$
        \chi (\lambda x) = \psi (\varphi \lambda x) = \psi (\lambda \varphi x) = \lambda\psi(\varphi x) = \lambda \chi x
    $$
\end{proof}
.\newline 
\theorem{(о матрице оператора-композиции операторов)}{Пусть $\chi = \psi \circ \varphi$, тогда $C_\chi = B_\psi \cdot A_\varphi$}
\begin{proof}
    Из определения следует: 
    $$
        \chi e_i = \psi (\varphi e_i) = \psi(\sum_{j = 1}^{m} \alpha^j g_j) = \sum_{j = 1}^{m} \alpha^j\psi(g_j) = \sum_{j = 1}^{m}\alpha^j(\sum_{k = 1}^{p} \beta^k_j h_k) = 
    $$
    $$
        = \sum_{k = 1}^{p}(\sum_{j = 1}^{m} \alpha_i^j \beta^k_j) h_k = \sum_{i = 1}^{p} \gamma^k_i h_k \quad \Rightarrow \quad \gamma^k_i = \sum_{j = 1}^{m} \alpha_i^j \beta^k_j
    $$
\end{proof}
\section{Алгебра линейных операторов и алгебра матриц}
\lemma{Операция композиции операторов ассоциативна}
    $$
        \varphi \in \text Hom (X, Y), \quad \psi \in  \text Hom (Y, Z), \quad \chi \in \text Hom (Z, W) 
    $$
\begin{proof}
    Покажем, что композиция ассоциативна всегда: 
    $$  
        (\chi \circ (\psi \circ \varphi))(x) = \chi ((\psi \circ \varphi)(x)) = \chi(\psi(\varphi(x))) = (\chi \circ \psi)(\varphi(x)) = ((\chi \circ \psi) \circ \varphi)(x)
    $$
\end{proof}
.\newline 
\notabene{Множество $\text{End}(X)$ имеет структуру полугруппы(ассоциативная операция) относительно операции композиции и структуру кольца относительно композиции и сложения}
\newline
\definition{\textbf{Алгеброй} называется кольцо, снабженное структурой линейного пространства.}
\newline
\definition{Алгебра End$(X, Y)$ называется \textbf{алгеброй операторов} над пространством $X(\mathbb{K})$.}
\subsection{Примеры алгебр}
\begin{enumerate}
    \item $\mathbb{R}$ - алгебра вещественных чисел;
    \item $\mathbb{C}$ - алгебра комплексных чисел.
\end{enumerate}
\subsection{Алгебра матриц}
\notabene{Очевидно, что каждому оператору можно сопоставить свою матрицу, а значит алгебра линейных операторов изоморфна алгебре квадратных матриц $n \times n$}
\newline 
\lemma{Имеет место изоморфизм алгебры ЛО и алгебры квадратных матриц $n \times n$}
$$
    \text End(X, Y) \simeq \text{Mat}_n
$$
\begin{proof}
    Выберем базис $\{ ^i_j \varepsilon \}$ в End$(X, Y)$ и отобразим
    $$
        \varphi = \sum_{i, j = 1}^{n} .^i_j \varepsilon \alpha^j_i \quad \leftrightarrow \quad ||\alpha^j_i|| = A_\varphi
    $$
\end{proof}
\section{Обратный оператор}
\definition{Обратным оператором к данному называется отображение $\bar \varphi \text{Im}\varphi \to X$, такое что}
$$
    \bar(y) = x \quad \forall y \in \text{Im}\varphi
$$
\lemma{Отображение $\bar{\varphi}$ - линейный оператор}
\begin{proof}
    Очевидно.
\end{proof}
.\newline
\definition{Оператор, для которого существует обратный, называется \textbf{обратимым}.}
\newline 
\definition{Линейный оператор $\varphi ^ {-1} \in$ Hom X Y называется \textbf{обратным оператором} к оператору $\varphi$, если}
$$
    (\bar{\varphi} \circ \varphi)(x) = x \quad \forall x \in X
$$
$$
    (\varphi \circ \bar{\varphi})(y) = y \quad \forall y \in Y
$$
\theorem{(критерий существования обратного оператора)}{Для оператора $\varphi \in$ Hom X Y существует обратный ему тогда и только тогда, когда}
$$
    \ker \varphi = {0}, \quad \text{Im}\varphi = Y
$$
\begin{proof}
    Первое - гарантирует инъективность отображения, а второе - сюръективность. Поэтому отображение является биекцией, а значит существует обратное.
\end{proof}
.\newline 
\notabene{Необходимым условием существования обратного оператора является изоморфность пространств $X$ и $Y$}
$$
    X \simeq Y \quad \Leftrightarrow \quad \dim X = \dim Y
$$
\lemma{Отображение $\varphi \mapsto \varphi^{-1}$ обладает следующими свойствами:}
$$
    (\varphi^{-1})^{-1} = \varphi, \quad (\psi \circ \varphi)^{-1} = \varphi^{-1} \circ \psi ^{-1}
$$
\begin{proof}
    Очевидно.
\end{proof}
\section{Обратимость в алгебре матриц}
\end{document}